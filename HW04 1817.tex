\documentclass{article}
\usepackage{amsmath}
\usepackage{enumitem}
\usepackage{amssymb}

\begin{document}
\hfill Hans Weeks and Christina Yu

\hfill Software Specifications

\hfill September 24, 2017
\section*{Homework 4}
In the following proof we show that program specified in assignment 3 meets the specification from assignment 2. Namely, that a matrix with 1 color and 1 label per element will eventually have each color-region with a matching label unique to that region. We do this by introducing 2 invariants, then showing that init leads to post using a well founded metric. Then we show that once we have reached post, the program does not change.\\

Invariant 1: $\langle \exists x: \pi(x) :: L(x) \neq \lambda(x) \ \rangle$ \\
Let $x$ = (i,j)  and $y$ = (p,q)\\
Initially, L(i, j) = (i, j). There is a label which is smaller than every other labels in the same region.\\
Furthermore, assume ``if\ SameRegion((i,j),(p,q))'' is true,\\
the statement L(i,j):= min(L(i,j),L(p,q)) ensures L(y)=$\lambda(y)$=$\lambda(x)$\\
Thus, there always exists a cell x whose label is $\lambda(x)$.\\
\\
Invariant 2: $\langle \forall x,y: \pi(x) \wedge \pi(y) \wedge \neg(x \Gamma y) :: L(x) \neq L(y) \ \rangle$ \\
Let $x$ = (i,j)  and $y$ = (a,b)\\
Initially, $L(x) = (i,j) \wedge L(y) = (a,b) \wedge (i,j)\neq(a,b) \implies L(x) \neq L(y)$\\
which means cell x and y have different initial labels.\\
Furthermore, there exists the statement $S_{ijpq}$ ensures L(x)=$\lambda(x)$ and\\a statement $S_{abcd}$ ensures L(y)=$\lambda(y)$.\\
$if \neg(x \Gamma y)$ is true, then $\lambda(x)$$\neq$$\lambda(y)$, $\implies$ L(x) $\neq$ L(y)\\
when L(x) $\neq$ L(y), none of the assignment alter the state.\\
\begin{enumerate}


\item
init $\leadsto$ post\\
let's define $\mu$:\\
$\mu = \langle \ \Sigma \ \forall \ x : \pi(x) :: L(x) \neq \lambda(x) \ \rangle\\$
This is well founded because initially only 1 element per region is correctly labeled.

With this $\mu$ in mind, we would like to show 2 things:\\
1. $\mu$ only decreases and 2. when $\mu=0$, then we have reached post.\\

Part 1:\\
\begin{align*}
&\mu = k \leadsto \mu < k
\end{align*}
$\mu$ increases or stays the same. If $k$ is 0, then it never decreases, just remains 0. If $k$ is greater than 0, then there exists a statement $S_{ijpq}$ where $(i,j)$ is an element that does not have its final label and $(p,q)$ does have its final label. Then\\
\begin{align*}
S_{ijpq} \ ensures \ \mu < k\\
\end{align*}
By invariant 1 we know that such an element pq always exists.\\

\iffalse
\begin{align*}
&\emph{Let's say that:}\\
&\pi(x)=(i,j) \wedge L(x)\neq \lambda(x) \wedge \pi(y)=(p,q) \wedge L(y)=\lambda(y)\\
&\emph{then}\\
&L(i,j) := min(L(i,j),L(p,q)) \ if \ SameRegion((i,j),(p,q)) \ ensures \ L(x)=\lambda(x)\\
&\emph{and}\\
&\langle \forall x \ : \pi(x)=(i,j) \ :: \exists \ i,j : 0 \leq i,j \leq N :: L(x)=\lambda(x) \ \rangle\\
\end{align*}
meaning that if x does not have its final label and y does have its final label, then we can select the assignment statement that takes x and y, and this statement ensures that $\mu$ decreases.\\
Additionally, there always exists such a $y$ that $L(y)=\lambda(y)$, because at least one element is set to the minimum label in its region.\\
\fi

Part2:\\
\begin{align*}
&\mu = 0 \implies post\\
\end{align*}

First, if the number of elements that don't have there final label is equal to zero, then all elements must have there final label\\
$\checkmark \ \mu = 0 \implies \langle \forall x : \pi(x) :: L(x)=\lambda(x) \rangle$\\

Second, if we have 2 elements that are not in the same region but have the same label, that implies that one of those elements does not have its final label:\\
\begin{align*}
&L(x) = L(y) \wedge \neg(x \Gamma y) \implies L(x)\neq\lambda(x) \vee L(y)\neq \lambda(y)\\
\end{align*}
which implies that
$\mu(x) > 0$\\
therefore\\
$\checkmark \mu(x)=0 \implies  \langle \forall x,y \ : \pi(x) \ \wedge \pi(x) \wedge \neg(x \Gamma y) :: \ L(x) \neq L(x)  \ \rangle\\$


\newpage
\item
\textbf{stable} post\\
\begin{align*}
&post \equiv \langle \forall x \ : \ \pi(x) \ :: \ L(x) = \lambda(x) \ \rangle\\
&\wedge \ \langle \forall x,y \ : \pi(x) \ \wedge \pi(x) \wedge \neg(x \Gamma y) :: \ L(x) \neq L(x)  \ \rangle\\
\\
&part 1:
post \equiv \langle \forall x \ : \ \pi(x) \ :: \ L(x) = \lambda(x) \ \rangle\\
&Let\ x = (i,j) \ and \ y = (p,q)\\
&There \ exists \ the \ statement\ S_{ijpq} \ ensures L(x)=\lambda(x) \ and\\ 
&a\ statement \ S_{abcd}\ ensures \ L(y)=\lambda(y).\\
&Assume\ ``if\ SameRegion((i,j),(p,q))'' \ is\ true,\ which\ means\ x\ \rho\ y.\\ 
&then \ \lambda(x)=\lambda(y)\ \implies L(x)=L(y)=\lambda(x)=\lambda(y)\\
&which\ means\ all\ cells\ in \ the\ same\ region\ share\ the\ same\ final\ label.\\
\\
&part 2:
\langle \forall x,y : \pi(x) \ \wedge \ \neg(x\Gamma y) :: L(x)\neq L(y) \rangle\\
&It\ is\ the\ invariant2\ proved\ on\ page\ 1.
\end{align*}
\end{enumerate}

\end{document}
