\documentclass{article}
\usepackage{amsmath}
\usepackage{enumitem}
\usepackage{amssymb}

\begin{document}
\hfill Hans Weeks and Christina Yu

\hfill Software Specifications

\hfill October 1, 2017
\section*{Homework 5: Formal properties}
\textbf{Program Description}\\
Consider a square matrix B of size N. Let A represent the initial configuration of the matrix B. \\
\begin{enumerate}
\item
Write a UNITY program that transposes the rows and columns of matrix B and preserves the following invariant:\\
\begin{align*}
 inv. \ p \leq \ q \wedge \\
&\langle \ \forall i,j \ : \ (1\leq i <p \vee 1 \leq j<p \vee q<i \leq N \vee q<j \leq N) \wedge \ 1 \leq i \leq N \wedge 1 \leq j \leq N \ :: \\ 
&B[i,j]=A[j,i]\rangle\\
\end{align*}
\\
\textbf{Program} Transpose\\
\\
\textbf{declare}\\
\begin{align*}
&A: array \ [1..N,\ 1..N]\ of \ integer\\
&p, q: integer
\end{align*}

\textbf{initially}\\
\begin{align*}
A=B\quad&\\
p=1\quad&\\
q=N\quad&\\
\end{align*}


\textbf{always}\\
\textbf{assign}\\
\begin{align*}
%&\langle \ || \ i \in\{p+1, \ q-1\} \wedge (q-1)< j < (p+1) \ ::  A[i,j]:=B[j,i]\rangle\\
%&||\\
%&\langle \ || \ j \in\{p+1, \ q-1\} \wedge (q-1)< i < (p+1) \ ::  A[i,j]:=B[j,i]\rangle\\
%&||\\
%&p,q := p+1,\ q-1\ if\ p \leq q
&p := p+1\ if \ p \leq q\\
& \quad \quad \sim q\\
&|| \ q := q-1 \ if \ p \leq q \\ 
&||\\
&\langle \ || \ j \in\{p,q\} \wedge q<i<p \ :: \ B[i,j]:=B[j,i] \rangle \\
&||\\
&\langle \ || \ i \in\{p,q\} \wedge q<j<p \ :: \ B[i,j]:=B[j,i] \rangle \\
\end{align*}


\item
Write a formal specification of the correctness of the program you designed. Such a specification often assumes the following general form:\\
\begin{align*}
&init \ \leadsto \ Post\\
&\textbf{stable} \ Post\\
&init \equiv A=B\\
&post \equiv \forall i,j : 1\leq i,j \leq N :: A[i,j]=B[j,i]\\
&\textbf{inv\ } 1\\
&\textbf{inv\ } \forall i,j : 1\leq i,j \leq N :: B[i,j]=A[i,j] \vee A[j,i] \\%prevents code ahead of p, behind q from being scrambled
\end{align*}
\\
\item
Explain in narrative form (no formal proof) the steps involved in proving these two properties.\\
\end{enumerate}



\end{document}
